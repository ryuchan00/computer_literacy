\usepackage{graphicx}
\usepackage{url}
\begin{document}
\title{コンピュータリテラシレポート#14}
\author{1920031 、山川竜太郎}
\date{2019/07/26}
\maketitle

\section{テーマ}
著作権に配慮した形で、独断と偏見でアメリカンヒーロー風にクラス、及び教師の紹介をする。脚色も十分に取り入れている。

\begin{itemize}
  \item 1920031 山川 竜太郎
  \item 1920003 伊東 隼人
  \item 1920005 院田 忠雅
  \item 1920033 渡辺 潔
\end{itemize}



http://www.edu.cc.uec.ac.jp/~y1920031/index.html

\section{グループ作業の内容}
2.1
役割分担\newline
\begin{itemize}
  \item Web 制作全体統括 : 山川
  \item キャラクターデザイン : 院田
  \item 画像の加工 : 院田、伊東
  \item 画面遷移案作成 : 伊東、渡邉
  \item CSS 設計 : 山川
  \item コーディング : 山川、伊東、渡邉、 ( 院田 )
  \item ファイル管理 : 山川
\end{itemize}

2.2
ファイル構成\newline
\begin{itemize}
  \item トップページ (index.html)
  \item 各ヒーローのページ (8 つ )
  \item デザイナーのページ
  \item 各種画像

\end{itemize}
これらを同一階層にて保存。 ( ファイル数がそんなに多くないため、階層化する必
要がないと判断 )\newline



2.3
制作にあたり ( 議論の前提 )\newline
以下のようなことを議論して、制作に取り掛かった。\newline
\begin{itemize}
  \item 我々は、テストがたくさんあり、制作にかけられる時間はほとんどない
ので、時間重視で制作することを共通認識としておいた。 ( その中で最大の
クオリティを目指す。 )
  \item テンプレートをつくり、そこにデータを当てはめる方向で制作を進める。\newline
 CSS の設計に多くの時間をかける。
  \item 前回つくったサイトのデータを活かすことで制作時間の短縮を目指す。
  \item デザイナーが描いてくれた画像 ( 一番のセールスポイント ) を最大限生かすこ
とが大事。 ( ヒーローデザインを元に全体のデザインをきめる。 )

\end{itemize}

\section{自分が担当したページの報告}
\begin{figure}[htbp]
\begin{center}
\includegraphics[width=3.8cm]{fig1.eps}
\caption{作成したページの表示です(かなり小さくなってしまっています)}\label{fig1}
\end{center}
\end{figure}


%\begin{center}\includegraphics[width=10cm]{fig1.eps}\end{center}
% PostScriptの絵を入れたければ上の行の「%」をはずしファイル名を修正

\section{考察}
CSSの割り振りは上から順に展開されていくことがわかった。所定の位置に、タグを持ってこれない場合はたいていCSSがずれていることが起因だと考えられる。見易いWebページを作成している方々はそれらを踏まえ作成されているのだと改めて認識しました。\newline

また、今回の課題は見出しタグのみのページまで作成し、そこから各自加工していった。そうすれば、最低限同じようなページができあがるだろうと考えていました。しかし、実際は見出しタグへ本文を書くパターンや、タグと本文を分けたりなどパターンがわかれてしまった。\newline

結果から、テンプレートを作成する際は見出しタグのほか、本文も適当なタグで囲いサンプル文章を用意することで更に作業効率を良いものにできると判断できると思いました。

\section{参考文献}
https://cccabinet.jpn.org/bootstrap4/example\newline
https://www.kanzaki.com/docs/html/htminfo16.html\newline
https://qiita.com/

\section{アンケート}

\subsection{Q1:Webサイトをグループで協力して製作してみて、どのようなことが分かりましたか。}
ぼくがWebページを起こせているのは仲間のみなさまのおかげです。今まで一度もやったことのなかった作業も丁寧に教えていただけたのでだんだんわかって来ました。自分でももっと深いところまでできるように調べながら勉強していきます。

\subsection{Q2:今回のようなレポートは何がよかったですか。何が大変でしたか。}
仲間と協力していいwebページができたと思っています。\newline
やはり、時間を合わせるのがたいへんでしたが、授業終りにダッシュで図書館に寄り相談したりして完成させました。

\subsection{Q3:リフレクション (今回の課題で分かったこと)・感想・要望をどうぞ。}
大変でしたが、完成させられて満足しています。



\end{document}v