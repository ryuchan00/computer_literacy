\documentclass[11pt,a4j]{jarticle}
\usepackage{graphicx}
\title{コンピュータリテラシレポート#09}
\author{1920005, 院田忠雅(ペア:1920031 山川竜太郎,1920003 伊東隼人)}
\date{20190616}
\begin{document}
\maketitle


\section{演習1a-c}
表示の数式をLatexで再現する。 

\section{レポートの本文}
再現します。
\begin{center}
  \begin{eqnarray*}
   a.  \, \, \,a_0+a_1+...+a_m\\
   b.  \,  \, \,\sum_{i=1}^m \int_0^i f(x)dx\\
 c. \, \, \,x = a_0+\frac{1}{a_1+\frac{1}{a_2+\frac{1}{a_3}}} 
  \end{eqnarray*}
\end{center}

\section{考察}
Latexを使用することで、数式を美しく表現することができる。\\
しかし、wordでも美しく表現できないまでも、マークアップ言語を介さず\\
簡単にほぼ同じことが出来るので、言語を覚える労力を他に回したほうが\\
ユーザーにとっては理にかなっている。\\
しかも、この言語は他のことには役に立たない。

\section{アンケート}

\subsection{Q1:普段どのくらい文書を作成していますか。またそのときに使うソフトや心がけていることなどを教えてください。}
Word,Pages,ノート等を使用してます。句読点や意味で終わる文節で改行するように心がけております。

\subsection{Q2:LaTeXのようなマークアップ型の文書作成系についてどのように感じましたか。}
少し、煩雑ですが使いこなせれば論文を書くことが可能となるので習得したい。

\subsection{Q3:リフレクション (今回の課題で分かったこと)・感想・要望をどうぞ。}
もっと使いこなして、論文を書きたい。

\end{document}
