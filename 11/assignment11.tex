\documentclass[12pt,a4j]{jarticle}
\usepackage{graphicx}
\begin{document}
\title{コンピュータリテラシレポート#11}
\author{1920031, 山川竜太郎(共同作業者  \and 1920005 院田忠雅 \and 1920003 伊東隼人 \and 1920007 江口航平 \and 1920033 渡辺潔)}
\date{2019/07/05}
\maketitle

\section{チームとして掲げたテーマ}
情報は無償であるべきではないか? 有償であってもよいか?

\section{討議の準備としてしたこと}
まずチームの議題に対して、「無償であるべき」「有償であるべき」とチーム分けした。私は「情報は無償であるべき」という立場から事前調査を行なった。まず行なったことは情報の定義を自分の中で決定することだった。漠然と情報と行っても幅が広いので討論するためには、定義を定めておく必要性がある。インターネットで調べた結果、私は情報を以下のように定義した。

「ある状況と別の状況との間に性質の相関があるとき、一方の状況が他方の状況について知る手がかりとなるもの」

次に自分がどの観点の情報に着手するか決める。この世で価格が無償で提供されている情報で私たちが普段利用しているものは何があるかと考えた場合、政府系のHPなどに記載されている情報が考えられる。具体的に言うと気象庁の災害情報などがこれにあたると考えられる。また無償で情報を入手する手段として、国立の図書館という手段も考えられる。これらを視野にいれると、政府はどのように無償提供されているか調査した。具体的には気象庁、経済産業省などがどのように速報を公表しているか調査して臨んだ。

\section{討議の内容}
まず無償派、有償派にチームを分ける。

無償派から出た意見を次の通り

\begin{itemize}
\item 学術情報、政府情報は国民は平等にアクセスできる権利がある。
\item 学術情報は信用をもとに作成されている。政府情報は信用できる情報でなければならない。それらは情報としての質は高いうえに、無償で提供されることにより国民に対して益をもたらす。
\end{itemize}

有償派から出た意見は次の通り

\begin{itemize}
\item 性悪説に基づくと、無償で情報提供すると一定数存在すると仮定すると、情報産業が成長せず、総合的に国益を損なう恐れがある。
\item 情報を無償にすると、収益がないので維持するコストを補うことができない。
\item 無償で情報を提供すると企業からの圧力がかかり、企業の存続が保てない可能性がある。コストに対して対価があるのが普通である。
\end{itemize}

議論をしたうえで、チームとして出た結論は以下のとおりである。

\begin{itemize}
\item 無償派と有償派の情報をまとめると、情報は受け取る人によって価値は異なる。例えば、アメリカに住んでいる人が東京の気象情報を受け取っても無価値であるが、東京に住んでいる人だと有益になりうる。価値が高い情報であれば有償でもいいし、価値が低い情報であれば無償でもよい。つまり需要と供給にのっとって価値を決めるべきだということ。
\item 情報そのものに対して価値を決めるのではなく、情報を使いやすいように加工(翻訳がこれにあたる)や抽出(新聞記事がこれにあたる)に対して有償であるべきである。
\end{itemize}

\section{結論}
まず学術情報や政府情報など、アカデミックやガバメントの情報は基本無償であるべきである。なぜなら人類規模でみれば大きい集合体である国家の運営、発展に際して必要なものであるためだ。基本的に情報がオープンな国家ほど経済的に豊かな傾向がある。例外もあるが資本主義国と社会主義国の差を見ればその傾向が強い。オープンな情報をもとに営利活動を行ったり、災害発生時に即座にその情報が国民に伝わるような仕組みを民間企業が補佐している。そのような情報は基本的に誰でもアクセスできる権利をもたせたほうが、その情報を直接受け取る人もいれば、加工抽出してより有益な形で提供することも可能である。それは国家の経済成長を支えるだろう。

企業が取得した情報については、有償で提供すべきだと思う。企業は営利団体であり、何かしらの収益を得て企業を存続している。情報の無償公開化を義務付けた場合に起こりうることは、情報の販売を事業としている場合、収益を上げられなくなり企業存続自体が困難になる。情報を価格は無償で提供するにしても広告などの収益モデルを採用しなければならない。つまり何かしら時間や金銭などの対価は必要であると考える。情報の加工抽出に対して何かしらの対価を払うべきである。ここまででまとめると、情報は無償のものと有償のものは使い分けるべきである。

\section{参考文献}

情報とは何か?

https://www.idia.jp/report/what-is-information

気象庁

https://www.jma.go.jp/jp/yoho/

経済産業省

https://www.jma.go.jp/jp/yoho/

\section{アンケート}

\subsection{Q1:調べる、討論する、考えるというプロセスはあなたにとってどのように有効でしたか。一 人で考えるのと比較して述べてください。}
私が知らない情報、思考プロセスを持っている人がいたので、一人で考えるよりも俯瞰して考えることができました。

\subsection{Q2:今回のようなレポートは何がよかったですか。何が大変でしたか。}
よかったことは、このレポート以外の成果物が不要だったことです。大変だったことはチームの情報共有はどのようにするのか、チームの議論をどのようにまとめるのか、結論をどのようにだすのかです。

\subsection{Q3:リフレクション (今回の課題で分かったこと)・感想・要望をどうぞ。}
情報について有償無償など、普段は考えなかったので、情報の価値について自分の中で考える機会になりました。

\end{document}
