\documentclass[12pt,a4j]{jarticle}
\usepackage{graphicx}
\begin{document}
\title{コンピュータリテラシレポート#11}
\author{1920031, 山川竜太郎(共同作業者 

\begin{itemize}
\item 1920005 院田忠雅 
\item1920003 伊東隼人
\item 1920007 江口航平
\item 1920033 渡辺潔
\end{itemize}

)}
\date{2019/07/05}
\maketitle

\section{チームとして掲げたテーマ}
情報は無償であるべきではないか? 有償であってもよいか?

\section{討議の準備としてしたこと}
まずチームの議題に対して、「無償であるべき」「有償であるべき」とチーム分けした。私は「情報は無償であるべき」という立場から事前調査を行なった。まず行なったことは情報の定義を自分の中で決定することだった。漠然と情報と行っても幅が広いので討論するためには、定義を定めておく必要性がある。インターネットで調べた結果、私は情報を以下のように定義した。

「ある状況と別の状況との間に性質の相関があるとき、一方の状況が他方の状況について知る手がかりとなるもの」

次に自分がどの観点の情報に着手するか決める。この世で価格が無償で提供されている情報で私たちが普段利用しているものは何があるかと考えた場合、政府系のHPなどに記載されている情報が考えられる。具体的に言うと気象庁の災害情報などがこれにあたると考えられる。また無償で情報を入手する手段として、国立の図書館という手段も考えられる。これらを視野にいれると、政府はどのように無償提供されているか調査した。具体的には気象庁、経済産業省などがどのように速報を公表しているか調査して臨んだ。

\section{討議の内容}
まず無償派、有償派にチームを分ける。

無償派から出た意見を次の通り

\begin{itemize}
\item 学術情報、政府情報は国民は平等にアクセスできる権利がある。
\end{itemize}

有償派から出た意見は次の通り

\begin{itemize}
\item 性悪説に基づくと、無償で情報提供すると一定数存在すると仮定すると、情報産業が成長せず、総合的に国益を損なう恐れがある。
\item 情報を無償にすると、収益がないので維持するコストを補うことができない。
\end{itemize}

\section{結論}
( 討論が終って自分がどのような結論に至ったかをその論拠・理由等も含めて筋道立てて書く )

\section{参考文献}

情報とは何か?

https://www.idia.jp/report/what-is-information

気象庁

https://www.jma.go.jp/jp/yoho/

経済産業省

https://www.jma.go.jp/jp/yoho/

\section{アンケート}

\subsection{Q1:調べる、討論する、考えるというプロセスはあなたにとってどのように有効でしたか。一 人で考えるのと比較して述べてください。}
( ここにQ1の回答を記入 )

\subsection{Q2:今回のようなレポートは何がよかったですか。何が大変でしたか。}
( ここにQ2の回答を記入 )

\subsection{Q3:リフレクション (今回の課題で分かったこと)・感想・要望をどうぞ。}
( ここにQ3の回答を記入 )

\end{document}
