\documentclass[12pt,a4j]{jarticle}
\usepackage{graphicx}
\begin{document}
\title{コンピュータリテラシレポート#11}
\author{1920031, 山川竜太郎(共同作業者 1920005 院田忠雅 1920003 伊東隼人 1920007 江口航平 1920033 渡辺潔)}
\date{2019/07/05}
\maketitle

\section{チームとして掲げたテーマ}
情報は無償であるべきではないか? 有償であってもよいか?

\section{討議の準備としてしたこと}
( ここに討論の前にどのようなことを考えたか、どのような資料を探して調べたか、その他どのような 準備をしたかを記入。)
まずチームの議題に対して、「無償であるべき」「有償であるべき」とチーム分けした。私は「情報は無償であるべき」という立場から事前調査を行なった。まず行なったことは情報の定義を自分の中で決定することだった。漠然と情報と行っても幅が広いので討論するためには、定義を定めておく必要性がある。インターネットで調べた結果、私は情報を以下のように定義した。

「ある状況と別の状況との間に性質の相関があるとき、一方の状況が他方の状況について知る手がかりとなるもの」



\section{討議の内容}
( 討論の内容(要約)を書く。)

\section{結論}
( 討論が終って自分がどのような結論に至ったかをその論拠・理由等も含めて筋道立てて書く )

\section{参考文献}

情報とは何か?

https://www.idia.jp/report/what-is-information

\section{アンケート}

\subsection{Q1:調べる、討論する、考えるというプロセスはあなたにとってどのように有効でしたか。一 人で考えるのと比較して述べてください。}
( ここにQ1の回答を記入 )

\subsection{Q2:今回のようなレポートは何がよかったですか。何が大変でしたか。}
( ここにQ2の回答を記入 )

\subsection{Q3:リフレクション (今回の課題で分かったこと)・感想・要望をどうぞ。}
( ここにQ3の回答を記入 )

\end{document}
